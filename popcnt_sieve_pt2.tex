%
% File:   popcnt_sieve.qasm
% Date:   28-Nov-18
% Author: E. W. Postlethwaite
%
% Simple illustration of a circuit we might want to implement during quantum sieving
%
%     defbox  add6,6,0,'\texttt{add}'
% 
%     qubit   v1
%     qubit   v3
%     qubit   a0
%     qubit   s^{01}_{0}
%     qubit   s^{23}_{0}
%     qubit   s^{01}_{1}
%     qubit   s^{23}_{1}
%     qubit   a2
%     qubit   u0
%     qubit   u1
%     qubit   u2
%     qubit   u3
% 
%     add6    a0,s^{01}_{0},s^{23}_{0},s^{01}_{1},s^{23}_{1},a2
% 
%     nop     v1

%  Time 01:
%    Gate 00 add6(a0,s^{01}_{0},s^{23}_{0},s^{01}_{1},s^{23}_{1},a2)
%    Gate 01 nop(v1)

% Qubit circuit matrix:
%
% v1: gAxA, n  
% v3: n  , n  
% a0: gAxC, n  
% s^{01}_{0}: gAxD, n  
% s^{23}_{0}: gAxE, n  
% s^{01}_{1}: gAxF, n  
% s^{23}_{1}: gAxG, n  
% a2: gAxH, n  
% u0: n  , n  
% u1: n  , n  
% u2: n  , n  
% u3: n  , n  

\documentclass[11pt]{article}
\input{xyqcirc.tex}

% definitions for the circuit elements

\def\gAxC{\gnqubit{\texttt{add}}{ddddd}\w\A{gAxC}}
\def\gAxD{\gspace{\texttt{add}}\w\A{gAxD}}
\def\gAxE{\gspace{\texttt{add}}\w\A{gAxE}}
\def\gAxF{\gspace{\texttt{add}}\w\A{gAxF}}
\def\gAxG{\gspace{\texttt{add}}\w\A{gAxG}}
\def\gAxH{\gspace{\texttt{add}}\w\A{gAxH}}
\def\gAxA{*-{}\w\A{gAxA}}

% definitions for bit labels and initial states

\def\bA{ \q{v_{1}}}
\def\bB{ \q{v_{3}}}
\def\bC{ \q{a_{0}}}
\def\bD{ \q{s^{01}_{0}}}
\def\bE{ \q{s^{23}_{0}}}
\def\bF{ \q{s^{01}_{1}}}
\def\bG{ \q{s^{23}_{1}}}
\def\bH{ \q{a_{2}}}
\def\bI{ \q{u_{0}}}
\def\bJ{ \q{u_{1}}}
\def\bK{ \q{u_{2}}}
\def\bL{ \q{u_{3}}}

% The quantum circuit as an xymatrix

\xymatrix@R=5pt@C=10pt{
    \bA & \gAxA &\n  
\\  \bB & \n   &\n  
\\  \bC & \gAxC &\n  
\\  \bD & \gAxD &\n  
\\  \bE & \gAxE &\n  
\\  \bF & \gAxF &\n  
\\  \bG & \gAxG &\n  
\\  \bH & \gAxH &\n  
\\  \bI & \n   &\n  
\\  \bJ & \n   &\n  
\\  \bK & \n   &\n  
\\  \bL & \n   &\n  
%
% Vertical lines and other post-xymatrix latex
%
}

\end{document}
